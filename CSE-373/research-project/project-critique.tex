\documentclass{article}
\usepackage[margin=.75in]{geometry}
\usepackage{hyperref}
\hypersetup{
  colorlinks=true
}

\begin{document}
\section{Critique}
\subsection{Previous Critique}
Deepfakes are becoming more common, and are becoming increasingly difficult to identify. Just the other day I saw a tiktok account that only had deepfake videos of Tom Cruise, and it was very believable. Only slight vocal intonations and a tiny bit of the uncanny valley tipped me off that something wasn’t quite right.

I for one, along with many others, tend to take more seriously what I can see with my own eyes than what someone else tells me. If I personally witness one thing, and someone else tells me something that contradicts that, I know who I’d believe. This approach used to be far more feasible, lending to the adage that “seeing is believing”. However, these days, and driven primarily by image manipulation and processing algorithms, things that we trust unquestioningly such as our own eyes, may prove to be less and less reliable.

\subsection{Expanded Critique}
According to the quadtree Wikipedia page, some common uses for quadtrees include:
\begin{itemize}
  \item Image processing
  \item Mesh generation
  \item Spatial indexing, point location queries, and range queries
  \item Efficient collision detection in 2d
\end{itemize}

All of these uses are common in facial recognition software. There has already been a large philosophical and ethical debate
on whether or not facial recognition software is even sometimes ethical, and if it is, in what circumstances.

According to studies cited by the AMA Journal of Ethics, facial recognition technology (FRT) has been used "to identify and
monitor [medical] patients, as well as to diagnose genetic, medical, and behavioral conditions."[1] FRT can also be used to
catch perpetrators of crime and identify individuals on terror watch lists. In addition to both of these, FRT can also
be used to find missing people, especially in combination with aging software. It can help to protect businesses against
theft by identifying known or suspected theives as they enter stores. It can also reduce the number of resources needed
for other authentication measures such as fingerprinting since it doesn't require physical contact or human interaction.

However, FRT also has negative ethical implications, which in my eyes, far outweigh the positive implications.
Many have likened the use of FRT by police forces to a 1984-esque dystopia, something I might call an "appeal to 1984": similar to an appeal to emotion.
In my opinion, even though appeals to 1984 are generally guilty of employing the slippery slope fallacy, I believe
that this is genuinely cause for concern.

My fears are not unfounded, though. FRT has already been used by authoritarian governments to indentify and prosecute peaceful protestors.
It also threatens privacy on an individual and societal level since there is no transparency as to what exactly is being done with the scans
after they are done. Countries with limited freedoms also use FRT to spy on and persecute citizens deemed to be trouble makers.
FRT also creates lots of issues with security since the databases used to store scans can be breached, and they already have [3].
There are also lots of dangers in potentially false positives. If two people look very similar, one may be prosecuted for the crimes
of the other, and with 7 Billion+ people on Earth, the odds of that are growing every day.

A great example of how FRT can be used to the detriment of the people and in service of tyranny is China's use in Hong Kong.
Civil rights protestors have been identified and arrested by CCP loyal Hong Kong police multiple times.

I believe that any improvement in this technology has the potential to be used in ethically questionable ways.
I don't want to contribute to the persecution of civil rights activists. I don't want to enable human rights abuses and authoritarian governments.
I do not have anything inherently against improving image processing software, and I do not think that our work
will necessarily realize my concerns, but I want us to take a careful and measured approach to our work.

I recommend that we continue with our work on the new image processing system, but dedicate substantial resources
to ensuring that we carry out our work in a careful and ethical manner, and are deliberative throughout
the process with ensuring that our technologies are used only for good.

\textbf{References}\newline
[1] \url{https://journalofethics.ama-assn.org/article/what-are-important-ethical-implications-using-facial-recognition-technology-health-care/2019-02}\newline
[2] \url{https://www.youtube.com/watch?v=RjmuEmUrJ4s}\newline
[3] \url{https://www.itpro.com/security/data-breaches/354866/clearview-ai-client-list-hacked}\newline
[4] \url{https://www.forbes.com/sites/zakdoffman/2019/08/26/hong-kong-exposes-both-sides-of-chinas-relentless-facial-recognition-machine/?sh=3dc471df42b7}\newline
[5] \url{https://www.nytimes.com/2019/07/26/technology/hong-kong-protests-facial-recognition-surveillance.html}\newline
[6] \url{https://en.wikipedia.org/wiki/Quadtree}
\end{document}
